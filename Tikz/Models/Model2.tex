\documentclass[tikz, 11pt]{standalone}

% Parameters

% Packages
\usepackage{amsmath,amssymb}
\usepackage[scaled]{helvet}
\renewcommand\familydefault{\sfdefault} 

% Font sizes
\newcommand{\tinySize}{\tiny}				% tiny corresponds to 6 pt
\newcommand{\textSize}{\scriptsize}		% scriptsize corresponds to 8 pt
\newcommand{\headingSize}{\small}		% small corresponds to 10 pt
\newcommand{\labelSize}{\large \bf}		% Large corresponds to 12 pt and bf for bold face

% Colors
\definecolor{discColor}{rgb}{1.0, 0.0, 0.0}
\definecolor{detColor}{rgb}{0.16, 0.16, 0.78}
\definecolor{onColor}{rgb}{0.37, 0.84, 0.0}
\definecolor{offColor}{rgb}{0.84, 0.12, 0.84}
\definecolor{psychoColor}{rgb}{1.0, 0.64, 0.02}
%\definecolor{modelColor}{rgb}{0.6, 0.25, 0.0}

\colorlet{retinaColor}{gray!25!white}
\colorlet{offEdgeColor}{offColor}
\colorlet{onEdgeColor}{onColor}
\colorlet{boxColor}{gray!20!white}
\newcommand{\fillSat}{40}
\usetikzlibrary{shapes.geometric, arrows.meta, snakes, calc, decorations.pathmorphing}

% Parameters
\newcommand{\unit}{1mm}
\newcommand{\pad}{0.5}
\newcommand{\objSize}{2.75}
\newcommand{\ySep}{0.3*\objSize}
\newcommand{\yShiftPathway}{1}
\newcommand{\yShiftBrain}{3}
\newcommand{\retinaWidth}{22}
\newcommand{\retinaHeight}{27}
\newcommand{\pathwayHeight}{9}
\newcommand{\sumHeight}{ (2*\objSize+2*\ySep) }
\newcommand{\brainWidth}{8}
\newcommand{\lineWidth}{0.5pt}
\newcommand{\cornerRadius}{0.2*\unit}
\newcommand{\arrow}{{Latex[length=\unit, width=\unit]}}
\pgfmathsetmacro{\xSepOff}{ (\retinaWidth-5*\objSize)/3 }%
\pgfmathsetmacro{\xSepOn}{ (\retinaWidth-5.5*\objSize)/4 }%


\colorlet{boxColor}{gray!20!white}
%\colorlet{boxColor}{white}

% Packages
\usepackage{amsmath,amssymb}
\usepackage[scaled]{helvet}
\renewcommand\familydefault{\sfdefault} 

% Styles
\tikzstyle{edge}=[draw=black, line width=\lineWidth, line cap=round, rounded corners=\cornerRadius]
\tikzstyle{edgeShort}=[edge, shorten < = 0.35*\unit, shorten > = 0.35*\unit]
\tikzstyle{edgeThick}=[edge, line width=2*\lineWidth]
\tikzstyle{photon}=[black, line width=\lineWidth, -\arrow, decorate, decoration={snake, segment length=1.25*\unit, amplitude=0.4*\unit, post length=1*\unit}]
\tikzstyle{textNode}=[inner sep=1mm, align=center, rounded corners=\cornerRadius, line width=1*\lineWidth]
\tikzstyle{textNodeThight}=[textNode, inner sep=0.5mm, outer sep=0]
\tikzstyle{textNodeLabel}=[textNodeThight,  text height=5pt, text depth=1pt]
\tikzstyle{box}=[rounded corners=\cornerRadius, line width=1*\lineWidth, outer sep=0]
\tikzstyle{obj}=[inner sep=0, draw=none, minimum width=\objSize*\unit, minimum height=\objSize*\unit, rounded corners=\cornerRadius, line width=1*\lineWidth]
%\tikzstyle{sumObj}=[obj, draw=black, regular polygon, regular polygon sides=3, shape border rotate=-90]
\tikzstyle{sumObj}=[obj, draw=none, minimum width=0.5*\objSize*\unit]
\tikzstyle{funObj}=[obj, draw=black, line width=1*\lineWidth]

% Prototypes
\tikzset{nonLinFun/.pic = {\draw[edgeThick] (-0.4*\objSize*\unit, -0.25*\objSize*\unit) --++ (0.4*\objSize*\unit, 0) --++ (0.4*\objSize*\unit, 0.5*\objSize*\unit);}}
\tikzset{linFun/.pic = {\draw[edgeThick] (-0.4*\objSize*\unit, -0.4*\objSize*\unit)  --++ (0.8*\objSize*\unit, 0.8*\objSize*\unit);}}
\tikzset{
	rod/.pic = {
		\node[rectangle, draw=none, fill=black, minimum width=0.8*\objSize*\unit, minimum height=0.1*\objSize*\unit,  anchor=west, rounded corners=\cornerRadius, inner sep=0, outer sep=0]  at (-0.5*\objSize, 0) {};
		\node[ellipse, draw=none, fill=black, minimum width=0.3*\objSize*\unit, minimum height=0.2*\objSize*\unit,  anchor=east, inner sep=0, outer sep=0]  at (0.5*\objSize, 0) {};}
}

\usetikzlibrary{shapes.geometric, arrows.meta, snakes, calc, decorations.pathmorphing}

% Parameters
\newcommand{\unit}{1mm}
\newcommand{\pad}{0.5}
\newcommand{\objSize}{2.75}
\newcommand{\ySep}{0.3*\objSize}
\newcommand{\yShiftPathway}{1}
\newcommand{\yShiftBrain}{3}
\newcommand{\retinaWidth}{22}
\newcommand{\retinaHeight}{27}
\newcommand{\pathwayHeight}{9}
\newcommand{\sumHeight}{ (2*\objSize+2*\ySep) }
\newcommand{\brainWidth}{8}
\newcommand{\lineWidth}{0.5pt}
\newcommand{\cornerRadius}{0.2*\unit}
\newcommand{\arrow}{{Latex[length=\unit, width=\unit]}}
\pgfmathsetmacro{\xSepOff}{ (\retinaWidth-5*\objSize)/3 }%
\pgfmathsetmacro{\xSepOn}{ (\retinaWidth-5.5*\objSize)/4 }%


\colorlet{boxColor}{gray!20!white}
%\colorlet{boxColor}{white}

% Packages
\usepackage{amsmath,amssymb}
\usepackage[scaled]{helvet}
\renewcommand\familydefault{\sfdefault} 

% Styles
\tikzstyle{edge}=[draw=black, line width=\lineWidth, line cap=round, rounded corners=\cornerRadius]
\tikzstyle{edgeShort}=[edge, shorten < = 0.35*\unit, shorten > = 0.35*\unit]
\tikzstyle{edgeThick}=[edge, line width=2*\lineWidth]
\tikzstyle{photon}=[black, line width=\lineWidth, -\arrow, decorate, decoration={snake, segment length=1.25*\unit, amplitude=0.4*\unit, post length=1*\unit}]
\tikzstyle{textNode}=[inner sep=1mm, align=center, rounded corners=\cornerRadius, line width=1*\lineWidth]
\tikzstyle{textNodeThight}=[textNode, inner sep=0.5mm, outer sep=0]
\tikzstyle{textNodeLabel}=[textNodeThight,  text height=5pt, text depth=1pt]
\tikzstyle{box}=[rounded corners=\cornerRadius, line width=1*\lineWidth, outer sep=0]
\tikzstyle{obj}=[inner sep=0, draw=none, minimum width=\objSize*\unit, minimum height=\objSize*\unit, rounded corners=\cornerRadius, line width=1*\lineWidth]
%\tikzstyle{sumObj}=[obj, draw=black, regular polygon, regular polygon sides=3, shape border rotate=-90]
\tikzstyle{sumObj}=[obj, draw=none, minimum width=0.5*\objSize*\unit]
\tikzstyle{funObj}=[obj, draw=black, line width=1*\lineWidth]

% Prototypes
\tikzset{nonLinFun/.pic = {\draw[edgeThick] (-0.4*\objSize*\unit, -0.25*\objSize*\unit) --++ (0.4*\objSize*\unit, 0) --++ (0.4*\objSize*\unit, 0.5*\objSize*\unit);}}
\tikzset{linFun/.pic = {\draw[edgeThick] (-0.4*\objSize*\unit, -0.4*\objSize*\unit)  --++ (0.8*\objSize*\unit, 0.8*\objSize*\unit);}}
\tikzset{
	rod/.pic = {
		\node[rectangle, draw=none, fill=black, minimum width=0.8*\objSize*\unit, minimum height=0.1*\objSize*\unit,  anchor=west, rounded corners=\cornerRadius, inner sep=0, outer sep=0]  at (-0.5*\objSize, 0) {};
		\node[ellipse, draw=none, fill=black, minimum width=0.3*\objSize*\unit, minimum height=0.2*\objSize*\unit,  anchor=east, inner sep=0, outer sep=0]  at (0.5*\objSize, 0) {};}
}


% Parameters
\newcommand{\nSubUnits}{6}

\begin{document}
\begin{tikzpicture}[x=\unit, y=\unit]

% Frame

\draw[white] (-3mm, 0mm) --++ (60mm, 0);

% Retina box
\node[box, draw=none, fill=none, minimum width=\retinaWidth*\unit, minimum height=\retinaHeight*\unit, anchor=west] (retina) at (5, 0) {};
% Retina texts
\node[textNodeThight, anchor=south] at (retina.north) {\textSize Retina};

% Brain box
\node[box, draw=none, fill=none , minimum width=\brainWidth*\unit, minimum height=\retinaHeight*\unit, anchor=west, xshift=2*\xSepOn*\unit+2.5*\objSize*\unit + \lineWidth +\pad*\unit] (brain) at (retina.east) {};
% Brain processing symbols
\newcommand{\tmpWidth}{5}
\node[obj, outer sep=0, fill=boxColor, minimum width=\brainWidth*\unit, minimum height=2*\pathwayHeight*\unit + 2*\yShiftPathway*\unit] (brainProcessing) at (brain) {};
%\fill[offColor!\fillSat!white] ($(brainProcessing.south west)+(0, \yShiftBrain)$) -- (brainProcessing.north west) -- (brainProcessing.north east) --++ (0, -\yShiftBrain) -- cycle;
%\fill[onColor!\fillSat!white] ($(brainProcessing.north east)+(0, -\yShiftBrain)$)  -- (brainProcessing.south east) -- (brainProcessing.south west) --++ (0, \yShiftBrain) -- cycle;
%\node[funObj, fill=onColor!\fillSat!white, anchor=south, xshift=-0*\unit, yshift=2*\unit] (brainNonlinear) at (brainProcessing.south) {};
%\node[funObj, fill=offColor!\fillSat!white, anchor=north, xshift=0*\unit, yshift=-2*\unit] (brainLinear) at (brainProcessing.north) {};
\node[funObj, fill=onColor!\fillSat!white, xshift=-0*\unit, yshift=-\yShiftPathway*\unit-0.5*\pathwayHeight*\unit] (brainNonlinear) at (brainProcessing) {};
\node[funObj, fill=offColor!\fillSat!white, xshift=0*\unit, yshift=\yShiftPathway*\unit+0.5*\pathwayHeight*\unit] (brainLinear) at (brainProcessing) {};
\draw[edgeShort] (brainLinear.west) --++ (-\xSepOn, 0);
\draw[edgeShort] (brainLinear.east) --++ (1.*\xSepOn, 0);
\draw[edgeShort] (brainNonlinear.west) --++ (-1.*\xSepOn, 0);
\draw[edgeShort] (brainNonlinear.east) --++ (\xSepOn, 0);
\pic[] at (brainNonlinear) {nonLinFun};
\pic[] at (brainLinear) {linFun};
\node[textNodeLabel, minimum width=\brainWidth*\unit, fill=onColor!\fillSat!white, anchor=north east] at (brainProcessing.south east) (nonlinLabelB) {\tinySize Nonlin.};
\node[textNodeLabel, minimum width=\brainWidth*\unit, fill=offColor!\fillSat!white, anchor=south west] at (brainProcessing.north west) (linLabelB) {\tinySize Linear};
\node[inner sep=0] (q) at (brain) {\textSize?};
%\draw[edge, shorten >=0.25*\unit] ($(brainProcessing.north east)+(0, -\yShiftBrain)$)  --  (q);
%\draw[edge, shorten <=0.25mm] (q) -- ($(brainProcessing.south west)+(0, \yShiftBrain)$);
% Brain texts
\node[textNodeThight, anchor=south] at (brain.north) {\textSize Brain};
% Outlines
%\draw[edge, onColor, shorten <=1.5*\unit, shorten >=1.5*\unit] ($(q)+(0, -1*\lineWidth)$) -- ($(brainProcessing.south west)+(0, \yShiftBrain)+(0, -1*\lineWidth)$) -- (brainProcessing.south west) -- (nonlinLabelB.north west) -- (nonlinLabelB.south west) -- (nonlinLabelB.south east) -- ($(brainProcessing.north east)+(0, -\yShiftBrain)+(0, -1*\lineWidth)$) -- ($(q)+(0, -1*\lineWidth)$);
%\draw[edge, offColor, shorten <=1.5*\unit, shorten >=1.5*\unit] ($(q)+(0, 1*\lineWidth)$) -- ($(brainProcessing.north east)+(0, -\yShiftBrain)+(0, 1*\lineWidth)$)  -- (brainProcessing.north east) -- (linLabelB.south east) -- (linLabelB.north east) -- (linLabelB.north west) -- ($(brainProcessing.south west)+(0, \yShiftBrain)+(0, 1*\lineWidth)$) -- ($(q)+(0, 1*\lineWidth)$) ;
\draw[edge, onColor, shorten <=1.5*\unit, shorten >=1.5*\unit] ($(q)+(0, -1*\lineWidth)$) -- ($(brainProcessing.west)+(0, -\yShiftBrain)+(0, -1*\lineWidth)$) -- (nonlinLabelB.south west) -- (nonlinLabelB.south east) -- ($(brainProcessing.east)+(0, -\yShiftBrain)+(0, -1*\lineWidth)$) -- ($(q)+(0, -1*\lineWidth)$);
\draw[edge, offColor, shorten <=1.5*\unit, shorten >=1.5*\unit] ($(q)+(0, 1*\lineWidth)$) -- ($(brainProcessing.east)+(0, \yShiftBrain)+(0, 1*\lineWidth)$)  -- (linLabelB.north east) -- (linLabelB.north west) -- ($(brainProcessing.west)+(0, \yShiftBrain)+(0, 1*\lineWidth)$) -- ($(q)+(0, 1*\lineWidth)$) ;

% Labels
\node[textNodeThight, anchor=north, xshift=2.25*\xSepOn*\unit, rotate=90] (behavior) at (brain.east) {\tinySize \color{black}  \textbf{Behavioral output}};
\draw[edgeShort, -\arrow]  (brain.east) -- (behavior.north);


% Name
\node[textNodeThight, anchor=south, text width=2.5cm, align=left, xshift=1.75*\objSize*\unit] at (retinaBox.north) {\baselineskip=6pt {\textSize Model 2 (M2)} \\{\tinySize Retinal~\&~brain~nonlin.\par}};
%\node[textNodeThight, anchor=south] at (retinaBox.north) {\textSize Model \textrm{II}};

% Brain nonlinearity
\node[box, fill=none, minimum width=\objSize*\unit+2*\pad*\unit, minimum height=\height*\unit, anchor=west, xshift=2*\xSep*\unit - 1*\pad*\unit] (brainBox) at (largeSum.east) {};
\node[funObj, fill=modelFillColor] (nonLinB) at (brainBox) {};
\pic[] at (nonLinB) {nonLinFun};
\draw[edgeShort] (largeSum.east) -- (nonLinB.west);
\draw[edgeShort] (nonLinB.east) --++ (\xSep, 0);



\end{tikzpicture}
\end{document}