\documentclass[tikz, 11pt]{standalone}

% Parameters

% Packages
\usepackage{amsmath,amssymb}
\usepackage[scaled]{helvet}
\renewcommand\familydefault{\sfdefault} 

% Font sizes
\newcommand{\tinySize}{\tiny}				% tiny corresponds to 6 pt
\newcommand{\textSize}{\scriptsize}		% scriptsize corresponds to 8 pt
\newcommand{\headingSize}{\small}		% small corresponds to 10 pt
\newcommand{\labelSize}{\large \bf}		% Large corresponds to 12 pt and bf for bold face

% Colors
\definecolor{discColor}{rgb}{1.0, 0.0, 0.0}
\definecolor{detColor}{rgb}{0.16, 0.16, 0.78}
\definecolor{onColor}{rgb}{0.37, 0.84, 0.0}
\definecolor{offColor}{rgb}{0.84, 0.12, 0.84}
\definecolor{psychoColor}{rgb}{1.0, 0.64, 0.02}
%\definecolor{modelColor}{rgb}{0.6, 0.25, 0.0}

\colorlet{retinaColor}{gray!25!white}
\colorlet{offEdgeColor}{offColor}
\colorlet{onEdgeColor}{onColor}
\colorlet{boxColor}{gray!20!white}
\newcommand{\fillSat}{40}
\usetikzlibrary{shapes.geometric, arrows.meta, snakes, calc, decorations.pathmorphing}

% Parameters
\newcommand{\unit}{1mm}
\newcommand{\pad}{0.5}
\newcommand{\objSize}{2.75}
\newcommand{\ySep}{0.3*\objSize}
\newcommand{\yShiftPathway}{1}
\newcommand{\yShiftBrain}{3}
\newcommand{\retinaWidth}{22}
\newcommand{\retinaHeight}{27}
\newcommand{\pathwayHeight}{9}
\newcommand{\sumHeight}{ (2*\objSize+2*\ySep) }
\newcommand{\brainWidth}{8}
\newcommand{\lineWidth}{0.5pt}
\newcommand{\cornerRadius}{0.2*\unit}
\newcommand{\arrow}{{Latex[length=\unit, width=\unit]}}
\pgfmathsetmacro{\xSepOff}{ (\retinaWidth-5*\objSize)/3 }%
\pgfmathsetmacro{\xSepOn}{ (\retinaWidth-5.5*\objSize)/4 }%


\colorlet{boxColor}{gray!20!white}
%\colorlet{boxColor}{white}

% Packages
\usepackage{amsmath,amssymb}
\usepackage[scaled]{helvet}
\renewcommand\familydefault{\sfdefault} 

% Styles
\tikzstyle{edge}=[draw=black, line width=\lineWidth, line cap=round, rounded corners=\cornerRadius]
\tikzstyle{edgeShort}=[edge, shorten < = 0.35*\unit, shorten > = 0.35*\unit]
\tikzstyle{edgeThick}=[edge, line width=2*\lineWidth]
\tikzstyle{photon}=[black, line width=\lineWidth, -\arrow, decorate, decoration={snake, segment length=1.25*\unit, amplitude=0.4*\unit, post length=1*\unit}]
\tikzstyle{textNode}=[inner sep=1mm, align=center, rounded corners=\cornerRadius, line width=1*\lineWidth]
\tikzstyle{textNodeThight}=[textNode, inner sep=0.5mm, outer sep=0]
\tikzstyle{textNodeLabel}=[textNodeThight,  text height=5pt, text depth=1pt]
\tikzstyle{box}=[rounded corners=\cornerRadius, line width=1*\lineWidth, outer sep=0]
\tikzstyle{obj}=[inner sep=0, draw=none, minimum width=\objSize*\unit, minimum height=\objSize*\unit, rounded corners=\cornerRadius, line width=1*\lineWidth]
%\tikzstyle{sumObj}=[obj, draw=black, regular polygon, regular polygon sides=3, shape border rotate=-90]
\tikzstyle{sumObj}=[obj, draw=none, minimum width=0.5*\objSize*\unit]
\tikzstyle{funObj}=[obj, draw=black, line width=1*\lineWidth]

% Prototypes
\tikzset{nonLinFun/.pic = {\draw[edgeThick] (-0.4*\objSize*\unit, -0.25*\objSize*\unit) --++ (0.4*\objSize*\unit, 0) --++ (0.4*\objSize*\unit, 0.5*\objSize*\unit);}}
\tikzset{linFun/.pic = {\draw[edgeThick] (-0.4*\objSize*\unit, -0.4*\objSize*\unit)  --++ (0.8*\objSize*\unit, 0.8*\objSize*\unit);}}
\tikzset{
	rod/.pic = {
		\node[rectangle, draw=none, fill=black, minimum width=0.8*\objSize*\unit, minimum height=0.1*\objSize*\unit,  anchor=west, rounded corners=\cornerRadius, inner sep=0, outer sep=0]  at (-0.5*\objSize, 0) {};
		\node[ellipse, draw=none, fill=black, minimum width=0.3*\objSize*\unit, minimum height=0.2*\objSize*\unit,  anchor=east, inner sep=0, outer sep=0]  at (0.5*\objSize, 0) {};}
}

\usetikzlibrary{shapes.geometric, arrows.meta, snakes, calc, decorations.pathmorphing}

% Parameters
\newcommand{\unit}{1mm}
\newcommand{\pad}{0.5}
\newcommand{\objSize}{2.75}
\newcommand{\ySep}{0.3*\objSize}
\newcommand{\yShiftPathway}{1}
\newcommand{\yShiftBrain}{3}
\newcommand{\retinaWidth}{22}
\newcommand{\retinaHeight}{27}
\newcommand{\pathwayHeight}{9}
\newcommand{\sumHeight}{ (2*\objSize+2*\ySep) }
\newcommand{\brainWidth}{8}
\newcommand{\lineWidth}{0.5pt}
\newcommand{\cornerRadius}{0.2*\unit}
\newcommand{\arrow}{{Latex[length=\unit, width=\unit]}}
\pgfmathsetmacro{\xSepOff}{ (\retinaWidth-5*\objSize)/3 }%
\pgfmathsetmacro{\xSepOn}{ (\retinaWidth-5.5*\objSize)/4 }%


\colorlet{boxColor}{gray!20!white}
%\colorlet{boxColor}{white}

% Packages
\usepackage{amsmath,amssymb}
\usepackage[scaled]{helvet}
\renewcommand\familydefault{\sfdefault} 

% Styles
\tikzstyle{edge}=[draw=black, line width=\lineWidth, line cap=round, rounded corners=\cornerRadius]
\tikzstyle{edgeShort}=[edge, shorten < = 0.35*\unit, shorten > = 0.35*\unit]
\tikzstyle{edgeThick}=[edge, line width=2*\lineWidth]
\tikzstyle{photon}=[black, line width=\lineWidth, -\arrow, decorate, decoration={snake, segment length=1.25*\unit, amplitude=0.4*\unit, post length=1*\unit}]
\tikzstyle{textNode}=[inner sep=1mm, align=center, rounded corners=\cornerRadius, line width=1*\lineWidth]
\tikzstyle{textNodeThight}=[textNode, inner sep=0.5mm, outer sep=0]
\tikzstyle{textNodeLabel}=[textNodeThight,  text height=5pt, text depth=1pt]
\tikzstyle{box}=[rounded corners=\cornerRadius, line width=1*\lineWidth, outer sep=0]
\tikzstyle{obj}=[inner sep=0, draw=none, minimum width=\objSize*\unit, minimum height=\objSize*\unit, rounded corners=\cornerRadius, line width=1*\lineWidth]
%\tikzstyle{sumObj}=[obj, draw=black, regular polygon, regular polygon sides=3, shape border rotate=-90]
\tikzstyle{sumObj}=[obj, draw=none, minimum width=0.5*\objSize*\unit]
\tikzstyle{funObj}=[obj, draw=black, line width=1*\lineWidth]

% Prototypes
\tikzset{nonLinFun/.pic = {\draw[edgeThick] (-0.4*\objSize*\unit, -0.25*\objSize*\unit) --++ (0.4*\objSize*\unit, 0) --++ (0.4*\objSize*\unit, 0.5*\objSize*\unit);}}
\tikzset{linFun/.pic = {\draw[edgeThick] (-0.4*\objSize*\unit, -0.4*\objSize*\unit)  --++ (0.8*\objSize*\unit, 0.8*\objSize*\unit);}}
\tikzset{
	rod/.pic = {
		\node[rectangle, draw=none, fill=black, minimum width=0.8*\objSize*\unit, minimum height=0.1*\objSize*\unit,  anchor=west, rounded corners=\cornerRadius, inner sep=0, outer sep=0]  at (-0.5*\objSize, 0) {};
		\node[ellipse, draw=none, fill=black, minimum width=0.3*\objSize*\unit, minimum height=0.2*\objSize*\unit,  anchor=east, inner sep=0, outer sep=0]  at (0.5*\objSize, 0) {};}
}


% Parameters
\newcommand{\nSubUnits}{6}

\begin{document}
\begin{tikzpicture}[x=\unit, y=\unit]

% Frame

\coordinate(lc) at (0, 0);
\node[box, fill=none, minimum width=\retinaWidth*\unit+2*\pad*\unit, minimum height=\height*\unit, anchor=west, xshift=-1*\pad*\unit] (retinaBox) at (lc) {};

\foreach \i in {1, 2, 3}{
	\pgfmathsetmacro{\yTmpSub}{ (\i-2)*(\objSize+\ySep) }
	% Rods
  	\foreach \j in {1,..., 3}{
  		\pgfmathsetmacro{\yTmpRod}{ 0.3*(\j-2)*(\objSize) }%
		\node[obj, outer sep=0, anchor=center, xshift=0.5*\objSize*\unit] (r\i\j) at ($(lc)+(0, \yTmpRod + \yTmpSub )$) {};
		\ifdim \i pt = 2pt
			\node[circle, fill=black, minimum width=1*\lineWidth, inner sep=0] at (r\i\j) {};
		\else
  			\pic[local bounding box=bbTmp] at (r\i\j) {rod};
		\fi
  	}
	% Subunits
	\ifdim \i pt = 2pt
		\node[anchor=center, inner sep=0.*\unit] (subunitCount) at (lc -| r\i1.west) {\tinySize n};
	\else
		\node[sumObj, xshift=\xSep*\unit + 1.25*\objSize*\unit] (subUnit\i) at (lc |- r\i2) {};
		\draw[edge, fill=modelFillColor] (subUnit\i.north west) -- (subUnit\i.east) -- (subUnit\i.south west) -- cycle;
		\node[funObj, fill=modelFillColor, xshift=2*\xSep*\unit + 2*\objSize*\unit] (nonLin\i) at (lc |- r\i2) {};
		\pic[] at (nonLin\i) {nonLinFun};
		% Noise
		\coordinate[xshift=\lineWidth] (tmp1) at (subUnit\i.east);
		\coordinate[yshift=-\objSize*\unit] (noise\i) at (tmp1);
		\pic[local bounding box=bbTmp] at (noise\i) {noiseAdd};
		\node[textNodeThight, anchor=east] (noiseAdd\i) at (bbTmp.west) {\tinySize N};
		\draw[edgeShort] (bbTmp.north) -- (tmp1);
	\fi
}

% Large sum
\node[obj, minimum height=\sumHeight*\unit, xshift=3*\xSep*\unit + 3*\objSize*\unit] (largeSum) at (lc) {}; 
\draw[edge, fill=modelFillColor] (largeSum.south west) -- (largeSum.north west) -- (largeSum.east) -- cycle;
\node[xshift=-0.2*\unit] at (largeSum) {\tinySize $\sum$};
\coordinate[xshift=\lineWidth] (tmp1) at (largeSum.east);
\coordinate[] (noiseMul) at (tmp1 |- noiseAdd1);
\pic[local bounding box=bbTmp] at (tmp1 |- noiseAdd1) {noiseMul};
\node[textNodeThight, anchor=east] (noiseMul1) at (bbTmp.west) {\tinySize N};
\draw[edgeShort] (bbTmp.north) -- (tmp1);

% Edges
\foreach \i in {1,3}{
	\foreach \j in {1,..., 3}{
		\draw[edgeShort] (r\i\j) -- (r\i\j -| subUnit\i.west);
	}
	\draw[edgeShort] (subUnit\i.east) -- (nonLin\i.west);
	\draw[edgeShort] (nonLin\i.east) -- (nonLin\i.east -| largeSum.west);
}


%\node[symObj] (noiseAddSym1) at (noiseAdd1.east -| subUnit1.east) {};
%\draw[edge, shorten > =0.25*\unit, shorten < =0.25*\unit] (noiseAddSym1.north) -- (noiseAddSym1.south); 
%\draw[edge, shorten > =0.25*\unit, shorten < =0.25*\unit] (noiseAddSym1.west) -- (noiseAddSym1.east); 
%\draw[edgeShort] (noiseAdd.north) -- (subUnit3.east);
%\node[symObj] (noiseAddSym3) at ($(noiseAdd.north)!0.5!(subUnit3.east)$) {};
%\draw[edge, shorten > =0.25*\unit, shorten < =0.25*\unit] (noiseAddSym3.north) -- (noiseAddSym3.south); 
%\draw[edge, shorten > =0.25*\unit, shorten < =0.25*\unit] (noiseAddSym3.west) -- (noiseAddSym3.east); 


% Name
\node[textNodeThight, anchor=south, text width=2.5cm, align=left, xshift=1.75*\objSize*\unit] at (retinaBox.north) {\baselineskip=6pt {\textSize Model 2 (M2)} \\{\tinySize Retinal~\&~brain~nonlin.\par}};
%\node[textNodeThight, anchor=south] at (retinaBox.north) {\textSize Model \textrm{II}};

% Brain nonlinearity
\node[box, fill=none, minimum width=\objSize*\unit+2*\pad*\unit, minimum height=\height*\unit, anchor=west, xshift=2*\xSep*\unit - 1*\pad*\unit] (brainBox) at (largeSum.east) {};
\node[funObj, fill=modelFillColor] (nonLinB) at (brainBox) {};
\pic[] at (nonLinB) {nonLinFun};
\draw[edgeShort] (largeSum.east) -- (nonLinB.west);
\draw[edgeShort] (nonLinB.east) --++ (\xSep, 0);



\end{tikzpicture}
\end{document}